
\documentclass[12pt,a4paper]{article}
\usepackage[utf8]{inputenc}
\usepackage[T1]{fontenc}
\usepackage{geometry}
\usepackage{fancyhdr}
\usepackage{url}
\usepackage{hyperref}

% Page setup
\geometry{margin=2.5cm}
\pagestyle{fancy}
\fancyhf{}
\rhead{\thepage}
\lhead{Ollama Flow Research Report}

\begin{document}

% Title Page
\begin{titlepage}
    \centering
    
    \vspace*{2cm}
    
    {\Huge\bfseries Ollama Flow}
    
    {\Huge\bfseries Multi-Role Research Report}
    
    \vspace{2cm}
    
    {\Large Automatisierte Recherche und Analyse}
    
    \vspace{3cm}
    
    \fbox{\begin{minipage}{0.8\textwidth}
        \centering
        \textbf{Anfrage:} Frauke Brosius-Gersdorf Hetzkampagne
    \end{minipage}}
    
    \vspace{2cm}
    
    \fbox{\begin{minipage}{0.6\textwidth}
        \centering
        \textbf{Vertrauenswert:} 29.7%
    \end{minipage}}
    
    \vfill
    
    {\large
    Erstellt am: 09.08.2025 20:36\\
    Workflow ID: 0cb9be9a-ceb4-4aaa-88bf-5b073b0a6991
    }
    
\end{titlepage}

\newpage
\tableofcontents
\newpage

\section{Recherche-Ergebnisse}

Die Recherche wurde von einem Team aus 3 spezialisierten Drohnen durchgeführt.

\subsection{Recherche-Task 1}

\textbf{Drohne:} researcher\_1\_enhanced\\
\textbf{Typ:} Enhanced\\
\textbf{Vertrauen:} 32.0%\\

\textbf{Erweiterte LLM-Recherche:}
\begin{itemize}
\item Frauke Brosius-Gersdorf war SPD-Kandidatin für das Bundesverfassungsgericht
\item Sie zog ihre Kandidatur nach einer intensiven öffentlichen Kampagne zurück
\item Die Kampagne richtete sich gegen ihre Positionen zu Schwangerschaftsabbrüchen und AfD-Verbot
\item Plagiatsvorwürfe gegen ihre Dissertation erwiesen sich als haltlos
\item Die SPD bezeichnete die Kampagne als 'beispiellos'
\end{itemize}

\textbf{Social Media Recherche:} 2 Quellen durchsucht

\textbf{Instagram:}
\begin{itemize}
\end{itemize}

\textbf{Web:}
\begin{itemize}
\item Wahl von Richtern des Bundesverfassungsgerichts 2025 - \url{https://de.wikipedia.org/wiki/Wahl_von_Richtern_des_Bundesverfassungsgerichts_2025}
\end{itemize}

\textbf{Suchzusammenfassung:} Durchsucht 5 Plattformen mit 2 Social Media Ergebnissen

\subsection{Recherche-Task 2}

\textbf{Drohne:} researcher\_2\_enhanced\\
\textbf{Typ:} Enhanced\\
\textbf{Vertrauen:} 32.0%\\

\textbf{Erweiterte LLM-Recherche:}
\begin{itemize}
\item Frauke Brosius-Gersdorf war SPD-Kandidatin für das Bundesverfassungsgericht
\item Sie zog ihre Kandidatur nach einer intensiven öffentlichen Kampagne zurück
\item Die Kampagne richtete sich gegen ihre Positionen zu Schwangerschaftsabbrüchen und AfD-Verbot
\item Plagiatsvorwürfe gegen ihre Dissertation erwiesen sich als haltlos
\item Die SPD bezeichnete die Kampagne als 'beispiellos'
\end{itemize}

\textbf{Social Media Recherche:} 2 Quellen durchsucht

\textbf{Instagram:}
\begin{itemize}
\end{itemize}

\textbf{Web:}
\begin{itemize}
\item Wahl von Richtern des Bundesverfassungsgerichts 2025 - \url{https://de.wikipedia.org/wiki/Wahl_von_Richtern_des_Bundesverfassungsgerichts_2025}
\end{itemize}

\textbf{Suchzusammenfassung:} Durchsucht 5 Plattformen mit 2 Social Media Ergebnissen

\subsection{Recherche-Task 3}

\textbf{Drohne:} researcher\_3\\
\textbf{Typ:} Traditional\\
\textbf{Vertrauen:} 30.0%\\

\textbf{Wichtige Fakten:}
\begin{itemize}
\item Frauke Brosius-Gersdorf war SPD-Kandidatin für das Bundesverfassungsgericht
\item Sie zog ihre Kandidatur nach einer intensiven öffentlichen Kampagne zurück
\item Die Kampagne richtete sich gegen ihre Positionen zu Schwangerschaftsabbrüchen und AfD-Verbot
\item Plagiatsvorwürfe gegen ihre Dissertation erwiesen sich als haltlos
\item Die SPD bezeichnete die Kampagne als 'beispiellos'
\end{itemize}

\textbf{Quellen:}
\begin{itemize}
\item Öffentliche Medienberichte
\item SPD-Pressemitteilungen
\item Verfassungsrechtliche Diskussionen
\end{itemize}

\textbf{Verschiedene Perspektiven:}
\begin{itemize}
\item SPD: Schmutzkampagne gegen qualifizierte Kandidatin
\item Union: Bedenken über politische Ausrichtung
\item Rechtswissenschaft: Diskussion über Richterauswahl
\item Medien: Kritik an Kampagnenführung
\end{itemize}

\textbf{Kontroversen und Diskussionspunkte:}
\begin{itemize}
\item Frage nach angemessener Kritik vs. Hetzkampagne
\item Rolle sozialer Medien bei Richterauswahl
\item Politisierung des Bundesverfassungsgerichts
\item Grenzen zulässiger öffentlicher Kritik
\end{itemize}

\newpage
\section{Konsolidierte Recherche-Ergebnisse}

\textit{Zusammenfassung aller einzigartigen Erkenntnisse aus der Multi-Drone-Recherche:}

\textbf{Wichtigste Fakten:}
\begin{enumerate}
\item Frauke Brosius-Gersdorf war SPD-Kandidatin für das Bundesverfassungsgericht
\item Sie zog ihre Kandidatur nach einer intensiven öffentlichen Kampagne zurück
\item Die Kampagne richtete sich gegen ihre Positionen zu Schwangerschaftsabbrüchen und AfD-Verbot
\item Plagiatsvorwürfe gegen ihre Dissertation erwiesen sich als haltlos
\item Die SPD bezeichnete die Kampagne als 'beispiellos'
\end{enumerate}

\textbf{Alle Quellen:}
\begin{itemize}
\item Öffentliche Medienberichte
\item SPD-Pressemitteilungen
\item Verfassungsrechtliche Diskussionen
\end{itemize}

\textbf{Verschiedene Perspektiven:}
\begin{itemize}
\item SPD: Schmutzkampagne gegen qualifizierte Kandidatin
\item Union: Bedenken über politische Ausrichtung
\item Rechtswissenschaft: Diskussion über Richterauswahl
\item Medien: Kritik an Kampagnenführung
\end{itemize}

\textbf{Kontroversen und Diskussionspunkte:}
\begin{itemize}
\item Frage nach angemessener Kritik vs. Hetzkampagne
\item Rolle sozialer Medien bei Richterauswahl
\item Politisierung des Bundesverfassungsgerichts
\item Grenzen zulässiger öffentlicher Kritik
\end{itemize}

\textbf{Social Media \& Web-Recherche:}
\textbf{Web:} (1 Ergebnisse)
\begin{itemize}
\item Wahl von Richtern des Bundesverfassungsgerichts 2025 - \url{https://de.wikipedia.org/wiki/Wahl_von_Richtern_des_Bundesverfassungsgerichts_2025}
\end{itemize}


\newpage
\section{Fact-Checking Ergebnisse}

Die Validierung wurde von einem spezialisierten Fact-Check-Team durchgeführt.

\fbox{\begin{minipage}{0.8\textwidth}
\textbf{Validierungsübersicht:} 0/2 Validierungen bestanden
\end{minipage}}

\subsection{Fact-Check 1}

\textbf{Drohne:} factchecker\_1\\
\textbf{Status:} ✗ Nicht bestanden\\

\begin{tabular}{ll}
\textbf{Plausibilität:} & 0/10 \\
\textbf{Logische Konsistenz:} & 0/10 \\
\textbf{Gesamtbewertung:} & 0/10 \\
\end{tabular}


\subsection{Fact-Check 2}

\textbf{Drohne:} factchecker\_2\\
\textbf{Status:} ✗ Nicht bestanden\\

\begin{tabular}{ll}
\textbf{Plausibilität:} & 0/10 \\
\textbf{Logische Konsistenz:} & 0/10 \\
\textbf{Gesamtbewertung:} & 0/10 \\
\end{tabular}



\newpage
\section{Datenanalyse}

Die finale Analyse wurde von spezialisierten Datenanalyse-Drohnen durchgeführt.

\subsection{Analyse 1}

\textbf{Drohne:} analyst\_1\\
\textbf{Vertrauen:} 0.0%\\

\subsection{Analyse 2}

\textbf{Drohne:} analyst\_2\\
\textbf{Vertrauen:} 0.0%\\


\newpage
\section{Fazit und Bewertung}

\fbox{\begin{minipage}{0.6\textwidth}
\centering
\textbf{Finaler Vertrauenswert:} 29.7%
\end{minipage}}

\textbf{Interpretation:} Geringe Vertrauenswürdigkeit - Weitere Recherche empfohlen.

\textbf{Workflow-Statistiken:}
\begin{tabular}{ll}
Recherche-Tasks: & 3 \\
Fact-Check-Tasks: & 2 \\
Analyse-Tasks: & 2 \\
Status: & completed \\
\end{tabular}

\subsection{Empfehlungen für weitere Schritte}

\begin{itemize}
\item Weitere Recherche mit zusätzlichen Quellen durchführen
\item Experten-Meinungen einholen
\item Spezifischere Suchanfragen formulieren
\end{itemize}



\end{document}
