
\documentclass[12pt,a4paper]{article}
\usepackage[utf8]{inputenc}
\usepackage[T1]{fontenc}
\usepackage{geometry}
\usepackage{fancyhdr}
\usepackage{url}
\usepackage{hyperref}

% Page setup
\geometry{margin=2.5cm}
\pagestyle{fancy}
\fancyhf{}
\rhead{\thepage}
\lhead{Ollama Flow Research Report}

\begin{document}

% Title Page
\begin{titlepage}
    \centering
    
    \vspace*{2cm}
    
    {\Huge\bfseries Ollama Flow}
    
    {\Huge\bfseries Multi-Role Research Report}
    
    \vspace{2cm}
    
    {\Large Automatisierte Recherche und Analyse}
    
    \vspace{3cm}
    
    \fbox{\begin{minipage}{0.8\textwidth}
        \centering
        \textbf{Anfrage:} irrelevante Quelle Test
    \end{minipage}}
    
    \vspace{2cm}
    
    \fbox{\begin{minipage}{0.6\textwidth}
        \centering
        \textbf{Vertrauenswert:} 23.7%
    \end{minipage}}
    
    \vfill
    
    {\large
    Erstellt am: 09.08.2025 22:14\\
    Workflow ID: 5aae5b5e-4d3c-41f5-a3fd-083c71ecbb5f
    }
    
\end{titlepage}

\newpage
\tableofcontents
\newpage

\section{Recherche-Ergebnisse}

Die Recherche wurde von einem Team aus 3 spezialisierten Drohnen durchgeführt.

\subsection{Recherche-Task 1}

\textbf{Drohne:} researcher\_1\_enhanced\\
\textbf{Typ:} Enhanced\\
\textbf{Vertrauen:} 36.0%\\

\textbf{Erweiterte LLM-Recherche:}
\begin{itemize}
\item Recherche zu: Grundlegende Fakten und Informationen zu: irrelevante Quelle Test
\item Wichtige Suchbegriffe: grundlegende, fakten, informationen, irrelevante, quelle
\item Hinweis: LLM-basierte Recherche nicht verfügbar, nutze Enhanced Mode für bessere Ergebnisse
\end{itemize}

\textbf{Social Media Recherche:} 6 Quellen durchsucht

\textbf{Twitter:}
\begin{itemize}
\item  Ungenaue und unvollständige Daten in den Meta-Elementen konnten so irrelevante Seiten bei spezifischen Suchen listen. Auch versuchten Seitenersteller… - \url{https://de.wikipedia.org/wiki/Suchmaschinenoptimierung}
\item der Muslimbruderschaft hatten über Twitter deutlich ihre Unterstützung für die Operation Decisive Storm on Twitter erklärt, gleichzeitig jedoch die ägyptische… - \url{https://de.wikipedia.org/wiki/Milit%C3%A4rintervention_im_Jemen_seit_2015}
\end{itemize}

\textbf{Instagram:}
\begin{itemize}
\end{itemize}

\textbf{Web:}
\begin{itemize}
\item Suchmaschinenoptimierung - \url{https://de.wikipedia.org/wiki/Suchmaschinenoptimierung}
\end{itemize}

\textbf{Suchzusammenfassung:} Durchsucht 5 Plattformen mit 6 Social Media Ergebnissen

\subsection{Recherche-Task 2}

\textbf{Drohne:} researcher\_2\_enhanced\\
\textbf{Typ:} Enhanced\\
\textbf{Vertrauen:} 36.0%\\

\textbf{Erweiterte LLM-Recherche:}
\begin{itemize}
\item Recherche zu: Historischer Kontext und Entwicklung von: irrelevante Quelle Test
\item Wichtige Suchbegriffe: historischer, kontext, entwicklung, von:, irrelevante
\item Hinweis: LLM-basierte Recherche nicht verfügbar, nutze Enhanced Mode für bessere Ergebnisse
\end{itemize}

\textbf{Social Media Recherche:} 6 Quellen durchsucht

\textbf{Twitter:}
\begin{itemize}
\item  Ungenaue und unvollständige Daten in den Meta-Elementen konnten so irrelevante Seiten bei spezifischen Suchen listen. Auch versuchten Seitenersteller… - \url{https://de.wikipedia.org/wiki/Suchmaschinenoptimierung}
\item der Muslimbruderschaft hatten über Twitter deutlich ihre Unterstützung für die Operation Decisive Storm on Twitter erklärt, gleichzeitig jedoch die ägyptische… - \url{https://de.wikipedia.org/wiki/Milit%C3%A4rintervention_im_Jemen_seit_2015}
\end{itemize}

\textbf{Instagram:}
\begin{itemize}
\end{itemize}

\textbf{Web:}
\begin{itemize}
\item Suchmaschinenoptimierung - \url{https://de.wikipedia.org/wiki/Suchmaschinenoptimierung}
\end{itemize}

\textbf{Suchzusammenfassung:} Durchsucht 5 Plattformen mit 6 Social Media Ergebnissen

\subsection{Recherche-Task 3}

\textbf{Drohne:} researcher\_3\\
\textbf{Typ:} Traditional\\
\textbf{Vertrauen:} 30.0%\\

\textbf{Wichtige Fakten:}
\begin{itemize}
\item Recherche zu: Aktuelle Trends und Zukunftsperspektiven zu: irrelevante Quelle Test
\item Wichtige Suchbegriffe: aktuelle, trends, zukunftsperspektiven, irrelevante, quelle
\item Hinweis: LLM-basierte Recherche nicht verfügbar, nutze Enhanced Mode für bessere Ergebnisse
\end{itemize}

\textbf{Quellen:}
\begin{itemize}
\item Lokale Wissensbasis
\item Suchbegriff-Analyse
\end{itemize}

\textbf{Verschiedene Perspektiven:}
\begin{itemize}
\item Themenbereich: Aktuelle Trends und Zukunftsperspektiven zu: irrelevante Quelle Test
\item Verschiedene Aspekte sollten durch Enhanced Mode abgedeckt werden
\end{itemize}

\textbf{Kontroversen und Diskussionspunkte:}
\begin{itemize}
\item Kontroversen erfordern tiefergehende Recherche
\end{itemize}

\newpage
\section{Konsolidierte Recherche-Ergebnisse}

\textit{Zusammenfassung aller einzigartigen Erkenntnisse aus der Multi-Drone-Recherche:}

\textbf{Wichtigste Fakten:}
\begin{enumerate}
\item Recherche zu: Aktuelle Trends und Zukunftsperspektiven zu: irrelevante Quelle Test
\item Wichtige Suchbegriffe: aktuelle, trends, zukunftsperspektiven, irrelevante, quelle
\item Hinweis: LLM-basierte Recherche nicht verfügbar, nutze Enhanced Mode für bessere Ergebnisse
\end{enumerate}

\textbf{Alle Quellen:}
\begin{itemize}
\item Lokale Wissensbasis
\item Suchbegriff-Analyse
\end{itemize}

\textbf{Verschiedene Perspektiven:}
\begin{itemize}
\item Themenbereich: Aktuelle Trends und Zukunftsperspektiven zu: irrelevante Quelle Test
\item Verschiedene Aspekte sollten durch Enhanced Mode abgedeckt werden
\end{itemize}

\textbf{Kontroversen und Diskussionspunkte:}
\begin{itemize}
\item Kontroversen erfordern tiefergehende Recherche
\end{itemize}

\textbf{Social Media \& Web-Recherche:}
\textbf{Twitter:} (2 Ergebnisse)
\begin{itemize}
\item  Ungenaue und unvollständige Daten in den Meta-Elementen konnten so irrelevante Seiten bei spezifischen Suchen listen. Auch versuchten Seitenersteller... - \url{https://de.wikipedia.org/wiki/Suchmaschinenoptimierung}
\item der Muslimbruderschaft hatten über Twitter deutlich ihre Unterstützung für die Operation Decisive Storm on Twitter erklärt, gleichzeitig jedoch die äg... - \url{https://de.wikipedia.org/wiki/Milit%C3%A4rintervention_im_Jemen_seit_2015}
\end{itemize}

\textbf{Web:} (1 Ergebnisse)
\begin{itemize}
\item Suchmaschinenoptimierung - \url{https://de.wikipedia.org/wiki/Suchmaschinenoptimierung}
\end{itemize}


\newpage
\section{Fact-Checking Ergebnisse}

Die Validierung wurde von einem spezialisierten Fact-Check-Team durchgeführt.

\fbox{\begin{minipage}{0.8\textwidth}
\textbf{Validierungsübersicht:} 0/2 Validierungen bestanden
\end{minipage}}

\subsection{Fact-Check 1}

\textbf{Drohne:} factchecker\_1\\
\textbf{Status:} ✗ Nicht bestanden\\

\begin{tabular}{ll}
\textbf{Thematische Relevanz:} & 5/10 \\
\textbf{Plausibilität:} & 0/10 \\
\textbf{Logische Konsistenz:} & 0/10 \\
\textbf{Gesamtbewertung:} & 3/10 \\
\end{tabular}

\textbf{Quellen-Glaubwürdigkeit:} unknown\\
\textbf{Potentielle Verzerrung:} not\_assessed\\
\textbf{Fehlende Informationen:}
\begin{itemize}
\item LLM-basierte Validierung nicht verfügbar
\end{itemize}

\subsection{Fact-Check 2}

\textbf{Drohne:} factchecker\_2\\
\textbf{Status:} ✗ Nicht bestanden\\

\begin{tabular}{ll}
\textbf{Thematische Relevanz:} & 5/10 \\
\textbf{Plausibilität:} & 0/10 \\
\textbf{Logische Konsistenz:} & 0/10 \\
\textbf{Gesamtbewertung:} & 3/10 \\
\end{tabular}

\textbf{Quellen-Glaubwürdigkeit:} unknown\\
\textbf{Potentielle Verzerrung:} not\_assessed\\
\textbf{Fehlende Informationen:}
\begin{itemize}
\item LLM-basierte Validierung nicht verfügbar
\end{itemize}


\newpage
\section{Datenanalyse}

Die finale Analyse wurde von spezialisierten Datenanalyse-Drohnen durchgeführt.

\subsection{Analyse 1}

\textbf{Drohne:} analyst\_1\\
\textbf{Vertrauen:} 0.0%\\

\subsection{Analyse 2}

\textbf{Drohne:} analyst\_2\\
\textbf{Vertrauen:} 0.0%\\


\newpage
\section{Fazit und Bewertung}

\fbox{\begin{minipage}{0.6\textwidth}
\centering
\textbf{Finaler Vertrauenswert:} 23.7%
\end{minipage}}

\textbf{Interpretation:} Geringe Vertrauenswürdigkeit - Weitere Recherche empfohlen.

\textbf{Workflow-Statistiken:}
\begin{tabular}{ll}
Recherche-Tasks: & 3 \\
Fact-Check-Tasks: & 2 \\
Analyse-Tasks: & 2 \\
Status: & completed \\
\end{tabular}

\subsection{Empfehlungen für weitere Schritte}

\begin{itemize}
\item Weitere Recherche mit zusätzlichen Quellen durchführen
\item Experten-Meinungen einholen
\item Spezifischere Suchanfragen formulieren
\end{itemize}



\end{document}
